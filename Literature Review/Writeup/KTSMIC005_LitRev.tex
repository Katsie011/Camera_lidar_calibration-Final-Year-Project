\documentclass[BCOR13mm,DIV15]{scrartcl}
% \documentclass[12pt]{IEEEtran}

% General document formatting
\usepackage[margin=0.7in, bottom=.7in]{geometry}
% \usepackage[parfill]{parskip}
\usepackage[utf8]{inputenc}

\usepackage[T1]{fontenc}
\usepackage{helvet}

\usepackage{graphicx}
\usepackage{wrapfig}
\usepackage{multicol}
\usepackage{caption}
\usepackage{subcaption}
\usepackage{amsmath}
\graphicspath{{../Figures/} {../Figures/Models/}{../Figures/PreProcessing/}}

\usepackage{lipsum}
\usepackage{natbib}

% \usepackage[
% backend=biber,
% style=alphabetic,
% citestyle=authoryear
% ]{biblatex}

% \addbibresource{EEE4022S.bib} %Imports bibliography file




\usepackage{xcolor}

%Dark mode pdf
% \pagecolor[rgb]{0.15,0.14,0.125} % Dark Grey
% \definecolor{myTextColour}{rgb}{0.9,0.9,0.8}

% Light Theme
\definecolor{myTextColour}{rgb}{0,0,0} 




\colorlet{thisTextColor}{myTextColour}
% \color{thisTextColor} % Whitish


%  CHange Link Colour!!!!!!!!!!!!!!!!



\usepackage{hyperref}
\hypersetup{
    colorlinks=true,
    linkcolor=red,
    filecolor=magenta,      
    urlcolor=blue,
}

\urlstyle{same}


\setlength{\parindent}{0pt}
\setlength{\parskip}{0.7em}






\begin{document}
\begin{titlepage}
    \setlength{\parindent}{0pt}
    \setlength{\parskip}{0pt}
    \vspace*{\stretch{1}}
    \begin{center}
        \vspace{-4cm}
        \huge University of Cape Town
    \end{center}
    \vspace{2.5cm}
    \begin{flushleft}
        \large \texttt{EEE4022S - Final Year Project}\\[8pt]
    \end{flushleft}

    \rule{\linewidth}{0.5pt}
    \begin{center}
    \vspace{10pt}
    \huge \textbf{Spacial and Temporal Calibration of Multi-Sensor Systems}\\[5pt]
    \rule{\linewidth}{0.5pt} \\[25pt]
    \LARGE Michael Katsoulis\\[5pt]
    \large \texttt{KTSMIC005} \\[1cm]
    24 August 2020\\ %                                Date
    % !!!!!!!!!!!!!!!!!!!!!!!!!!!!!!!!!!!!!!Check Date!!!!!!!!!!!!!!!!!!!!!!!!!!!
    \rule{5cm}{0.5pt}\\[-5pt]
    \rule{2.7cm}{0.5pt}
    \end{center}
    \vspace*{\stretch{1}}
    \Large{ \textbf{Plagerism Declaration}}

    \hrulefill
    \normalsize
    \begin{enumerate}
        \item I  know that  plagiarism is wrong. Plagiarism  is  to use  another's  work  and  pretend that  it is one's own.
        \item I  have  used  the IEEE convention  for  citation  and  referencing.  Each  contribution  to,  and quotation  in,  this final  year  project  report from  the  work(s)  of  other  people,  has  been attributed and has been cited and referenced.
        \item This project report is my own work.
        \item I have not allowed, and will not allow, anyone to copy my work with the intention of passing it off as their own work or part thereof.
    \end{enumerate}
    \begin{flushright}
        Michael Katsoulis\\
        24 August 2020
    \end{flushright}
    \hrulefill
\end{titlepage}
% ---------------------------------------------------------------------------
% check the date

\newpage
%%%%%%%%%%%%%%%%%%%%%%%%%%%%%%%%%%%%%%%%%%%%%%%%%%%%%%%%%%%%%%%%%%%%%%%%%%%%%%%%


% \tableofcontents
% \pagebreak


\section{Literature Review}
\subsection{Introduction to the problem}


In the modern world, the need for autonomous vehicles is growing daily. If common vehicles could become autonomous, it would improve the quality of human life, provide advances in science and open new opportunities. One of the problems with trying to make vehicles autonomous is that they are normally different shapes and sizes. Vehicles are sometimes modified or have accidents which would deform them in some way. This leads to the sensors on each vehicle being in different locations.  

The goal of this project is to investigate the current methods of determining the spacial and temporal offsets between sensors on a vehicle automatically. This would allow a decrease in calibration time, ease of manufacture as well as allow vehicles to continue to operate in the case of an unforeseen incident which moves a sensor.  

This would save costs as the same algorithm on multiple models of vehicle. Drones could be built which can reconfigure their geometry mid-flight or ,if damaged on a mission, recalibrate and still get home safely.

How do we get each form of error
Spacial --> straight forward
Temporal --> filters, length of communication, different clocks in sensors.


Attempts to fix temporal error
- clock synchronisation
- sensor reading time stamping 
etc. 


From the literature that has been reviewed, there are two main categories of calibration methods. The first uses markers that are of known shape, texture and size by design which are placed into the environment and used for calibration. This is a more hands-on, involved approach but it is often simpler. 
The other method is markerless calibration. In these cases, the goal is to use the information in the natural environment around the vehicle in order to calibrate the sensors. This often involves removing targets which are moving relative to the global frame and trying to isolate those which are static. 

The most common sensors used in autonomous vehicles are lidar and stereo cameras. 
On the fly Camera \& LiDAR Calibration
TODO ref
The main task is to find an efficient way to map point clouds from lidar sensors to the pixels of camera sensors. In 
TODO ref
the solution to this problem was to have two sets of calibration, one rough and the other small detail fine alignment. To do the coarse alignment, they expanded the point dots into planes and tried to find objects in both the lidar and camera images using straight lines.

-- CNNs for finding these objects?





\bibliography{EEE4022S.bib}
\bibliographystyle{abbrv}
 
% \printbibliography
\end{document}